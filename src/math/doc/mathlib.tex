\documentclass[notitlepage,a4paper,dvipdfm]{article}
\usepackage[latin1]{inputenc}
\usepackage{color}
\usepackage{epsfig}
\usepackage{listings}

\newcommand{\template}[2]{\textbf{#1$<$#2$>$}}
\newcommand{\templaten}[1]{\textbf{#1}}
\newcommand{\tpar}[1]{\textbf{#1}}

\newcommand{\concept}[2]{\textit{#1$<$#2$>$}}
\newcommand{\conceptn}[1]{\textit{#1}}

\newcommand{\keyw}[1]{\textsl{#1}}
\newcommand{\func}[1]{\textbf{#1}}
\newcommand{\const}[1]{\textbf{#1}}
\newcommand{\class}[1]{\textbf{#1}}
\newcommand{\id}[1]{#1}

\newcommand{\R}{\mathrm R}
\newcommand{\N}{\mathrm N}
\newcommand{\I}{\mathrm I}

\newcommand{\bref}[1]{[\ref{#1}]}

\newcommand{\mathlib}{{\sf math2}}
\newcommand{\mathlibr}{$\mathlib_\mathit{dr1\alpha}$}

\lstdefinelanguage{mathlib_conecepts}
	{morekeywords={static,bool,int,float,template,void,class,const,
		struct,concept,typedef,operator,true,false,explicit,union},
	sensitive,}[keywords]
\newcommand{\Matrix}{\conceptn{Matrix}}
\newcommand{\Vector}{\conceptn{Vector}}

\lstset{language=mathlib_conecepts}
\lstset{keywordstyle=\keyw}
\lstset{escapechar=!}
\lstset{mathescape=true}
\lstset{breaklines=true}

\title{math2}
\author{Daniel Wessl�n}

\begin{document}

{\center
	{\huge\mathlibr{} documentation}\\
	{\small graphlib.sourceforge.net}\\
}
\vspace{1cm}

\tableofcontents
%\newpage

\section{Introduction}
\label{intro}

\mathlib{} is a free (public domain) mathematics library, specializing
in high speed calculations for small matrices.
\\
\\
\noindent
A lot of this document contains information that is not neccesary if all you 
want to do is use \mathlib{} for some calculations, so feel free to skip ahead.
It is however recommended that you read the chapters Introduction \bref{intro},
Templates \bref{templates}, Functions \bref{functions},
Constants \bref{constants} and Examples \bref{examples} before using
\mathlib{}.
Doing so will help you avoid reimplementing something that can already
be done using \mathlib{}.

Several different notations are in use for matrices, the rest of this 
docuentation describes matrices and vectors as used in \mathlib, 
to avoid confusion.

\subsection{Matrices}
\index{Matrix}
A matrix with 4 columns and 2 rows is called a $4 \times 2$ matrix.
The element at row 3, column 1 in the matrix $A$ is $A_{31}$.

\[A_{2 \times 3} = A_{ij} = 
\left[\begin{array}{ccc}
	A_{11} & A_{12} & A_{13} \\
	A_{21} & A_{22} & A_{23}
\end{array}\right]\]

The primary representation of a matrix in \mathlib{} is
\templaten{matrix} \bref{matrix}.
While element indices start at one in this documentation, because
that is the normal mathematical notation, they start at zero in C++.

The set of fully supported matrix sizes is $r \times c$ where
$(r,c) \in [1,4]^2$.
Matrices with one side larger than 4 are supported for some operations,
in particular most elementwise operations are supported.
$1 \times 1$ matrices behave like scalars.

\subsection{Vectors}
\index{Vector}
A vector is a column matrix, meaning that a $\R^d$ vector is a
$d \times 1$ matrix and can be used wherever such a matrix is expected.

\[v \in \R^3 = v_i = 
\left[\begin{array}{c}
	v_{1} \\
	v_{2} \\
	v_{3} 
\end{array}\right]
=
\left[\begin{array}{c}
	v_{11} \\
	v_{21} \\
	v_{31} 
\end{array}\right]
= \left( v_1, v_2, v_3 \right)^{\rm t}\]

To increase readability, $\R^4$, $\R^3$ and $\R^2$ vectors have named elements.
\[v_{\rm xyzw} = v_{\rm stuv} = v_{\rm rgba}\]
\[v_{\rm xyz} = v_{\rm stu} = v_{\rm rgb}\]
\[v_{\rm xy} = v_{\rm st}\]

The primary representation of a vector in \mathlib{} is
\templaten{vector} \bref{vector}.

\subsection{Concepts}
\index{Concept}
A concept is a set of requirements.
Any class that meets the requirements is said to implement the concept.

The arguments to most functions in \mathlib{} are required to be instances
of specific concepts rather than classes, this is beacuse of the expression 
template technique described in \bref{exprtempl}.

It is usually ok to think of a \conceptn{Matrix} as a \templaten{matrix},
a \conceptn{Vector} as a \templaten{vector}, etc.
Concepts are described in more detail in \bref{concepts}.


\section{Templates}
\label{templates}

\subsection{Matrix}
\label{matrix}
\index{Matrix}
\begin{lstlisting}{matrix_t}
class !\template{matrix}{Rows, Cols = Rows}!
	: !\template{tmatrix}{\template{basic\_matrix}{Rows, Cols}}! {
	!\templaten{matrix}!(!\concept{Matrix}{Rows, Cols}!);
	!\templaten{matrix}!(!\conceptn{Initializer}!);
	!\templaten{matrix}!();

	!Member operators are described in \bref{arithmetic}.!
}
\end{lstlisting}


\subsection{Vector}
\label{vector}
\index{Vector}
\begin{lstlisting}{matrix_t}
class !\template{vector}{Rows}!
	: !\template{tmatrix}{\template{basic\_matrix}{Rows, 1}}! {
	!\templaten{vector}!(!\concept{Vector}{Rows}!);
	!\templaten{vector}!(!\conceptn{Initializer}!);
	!\templaten{vector}!();

	explicit !\templaten{vector}!(float);
	!Fills the vector.!

	!\templaten{vector}!(float, float);
	!\templaten{vector}!(float, float, float);
	!\templaten{vector}!(float, float, float, float);
	!Only the constructor with the same number of elements as the number of rows actually work.!

	!Member operators are described in \bref{arithmetic}.!
}
\end{lstlisting}

\subsection{Basic\_matrix}
\label{basic_matrix}
The template \template{basic\_matrix}{Rows, Cols} provides storage for
\templaten{matrix} and \templaten{vector}.
%
It does not implement any of the major concepts but provides the required 
members (\conceptn{MatrixBase} \bref{matrix_base_c}) for use as a base class to 
\templaten{tmatrix} for implementing \conceptn{Matrix} or \conceptn{Vector}.

\begin{lstlisting}{basic_matrix_t}
class !\template{basic\_matrix}{Rows, Cols}! {
	float union { x, s, r }
	float union { y, t, g }
	float union { z, u, b }
	float union { w, v, a }
	!Named vector elements. Only exists when \tpar{Cols} = 1 and the number of rows is suitable.!

	float &operator()(int r, int c);
	float operator()(int r, int c) const;
	!Element access.!

	float &operator()(int r);
	float operator()(int r) const;
	!Element access. For vectors only.!

	float *data();
	const float *data() const;
	!Elements in column major order.!

	!\template{matrix}{Rows, Cols}! &matrix()
	const !\template{matrix}{Rows, Cols}! &matrix() const
	!Casts the object to \templaten{matrix}.!

	void assign(float, float);
	void assign(float, float, float);
	void assign(float, float, float, float);
	!Assignment for vectors. Only the one with the same number of elements as the number of rows exists.!
}
\end{lstlisting}

\section{Functions}
\label{functions}

\subsection{Arithmetic}
\label{arithmetic}

\begin{lstlisting}{}
!\concept{Matrix}{R,C}! !\func{operator+}!(!\concept{Matrix}{R,C}!, !\concept{Matrix}{R,C}!)
!\concept{LMatrix}{R,C}! !\func{operator+=}!(!\concept{LMatrix}{R,C}!, !\concept{Matrix}{R,C}!)
	!Memberwise addition. $A + B$.!

!\concept{Matrix}{R,C}! !\func{operator-}!(!\concept{Matrix}{R,C}!, !\concept{Matrix}{R,C}!)
!\concept{LMatrix}{R,C}! !\func{operator-=}!(!\concept{LMatrix}{R,C}!, !\concept{Matrix}{R,C}!)
	!Memberwise subtraction. $A + B$.!

!\concept{Matrix}{R,C}! !\func{operator-}!(!\concept{Matrix}{R,C}!)
	!Unary negate. $-A$.!

!\concept{Matrix}{A,C}! !\func{operator*}!(!\concept{Matrix}{A,B}!, !\concept{Matrix}{B,C}!)
!\concept{LMatrix}{A,A}! !\func{operator*=}!(!\concept{LMatrix}{A,A}!, !\concept{Matrix}{A,A}!)
	!Matrix multiplication. $AB$.!

!\concept{Matrix}{R,C}! !\func{operator*}!(!\concept{Matrix}{R,C}!, float)
!\concept{Matrix}{R,C}! !\func{operator*}!(float, !\concept{Matrix}{R,C}!)
!\concept{LMatrix}{A,C}! !\func{operator*=}!(!\concept{LMatrix}{A,B}!, float)
	!Memberwise multiplication. $Aa$, $aA$.!

!\concept{Vector}{R}! !\func{operator*}!(!\concept{Vector}{R}!, !\concept{Vector}{R}!)
!\concept{LVector}{R}! !\func{operator*=}!(!\concept{LVector}{R}!, !\concept{Vector}{R}!)
	!Memberwise multiplication. $v_i u_i$.!

!\concept{Matrix}{R,C}! !\func{operator/}!(!\concept{Matrix}{R,C}!, float)
!\concept{LMatrix}{A,C}! !\func{operator/=}!(!\concept{LMatrix}{A,B}!, float)
	!Memberwise division. $v_i / a$.!

!\concept{Vector}{R}! !\func{operator/}!(!\concept{Vector}{R}!, !\concept{Vector}{R}!)
!\concept{LVector}{R}! !\func{operator/=}!(!\concept{LVector}{R}!, !\concept{Vector}{R}!)
	!Memberwise division. $v_i / u_i$.!

!\concept{Vector}{3}! !\func{cross}!(!\concept{Vector}{3}!, !\concept{Vector}{3}!)
	!Vector product. $v \times u$.!

float !\func{dot}!(!\concept{Vector}{R}!, !\concept{Vector}{R}!)
	!Scalar product. $v \cdot u$, $v^{\rm t} u$.!

!\concept{Vector}{R}! !\func{reciprocal}!(!\concept{Vector}{R}!)
	!Elementwise reciprocal. $1 / v_i$.!

float !\func{abs}!(!\conceptn{Vector}!)
	!The absolute value of $v$. $|v|$, $\sqrt{v \cdot v}$.!

float !\func{abs\_squared}!(!\conceptn{Vector}!)
	!The squared absolute value of $v$. $|v|^2$, $v \cdot v$.!

!\concept{Vector}{R}! !\func{norm}!(!\concept{Vector}{R}!)
	!The norm of $v$. $v / |v|$.!

float !\func{det}!(!\concept{Matrix}{A,A}!)
float !\func{det}!(!\concept{Vector}{2}!, !\concept{Vector}{2}!)
float !\func{det}!(!\concept{Vector}{3}!, !\concept{Vector}{3}!, !\concept{Vector}{3}!)
float !\func{det}!(float $x_1$, float $y_1$, float $x_2$, float $y_2$)
float !\func{det}!(float $x_1$, $y_1$, $z_1$, $x_2$, $y_2$, $z_2$, $x_3$, $y_3$, $z_3$)
	!Determinant. $\det A$.!

!\concept{Matrix}{A,A}! !\func{inverse}!(!\concept{Matrix}{A,A}!)
	!Inverse. $A^{-1}$.!

!\concept{Matrix}{C,R}! !\func{transpose}!(!\concept{Matrix}{R,C}!)
	!Transpose. $R_{ij} = A_{ji}$. $A^{\rm t}$.!

!\concept{Matrix}{R-1,C-1}! !\func{cofactor}$<i$,$j>$!(!\concept{Matrix}{R,C}!)
	!Remove row $i$ and column $j$ form $A$. $\mathrm{cofactor} \; A_{ij}$!

!\concept{Matrix}{C,R}! !\func{adjoint}!(!\concept{Matrix}{R,C}!)
	!Adjoint. $R_{ij} = (-1)^{i+j} \; \det \mathrm{cofactor} \; A_{ji}$. Currently only implemented for square matrices.!

!\concept{Vector}{3}! !\func{wdiv}!(!\concept{Vector}{4}!)
	!Perspective division. Converts from homogenous coordinates to cartesian space. $v_{\rm xyz} / v_{\rm w}$.!

!\template{matrix}{R,C}! !\func{temp}!(!\concept{Matrix}{R,C}!)
	!Evaluate to a temporary variable. Workaround for aliasing expressions.!

\end{lstlisting}


\subsection{Indexing}
Indexing transformations are described in detail in \bref{indexing}.

\begin{lstlisting}{}
!\conceptn{Indexer}! !\conceptn{Indexer}!::!\func{operator,}!(!\conceptn{Indexer}!)
	!Concatenate indexing transformations.!

!\conceptn{Indexer}! !\conceptn{Indexer}!::!\func{operator()}!(!\conceptn{Indexer}!)
	!Concatenate indexing transformations. Apply argument first.!

!\conceptn{Matrix}! !\conceptn{Matrix}!::!\func{operator[]}!(!\conceptn{Indexer}!)
	!Generic indexing.!

\end{lstlisting}

\subsection{IO and assignment}

\begin{lstlisting}{}
!\conceptn{LMatrix} $<<$! $a, b, c, d, e \dots$
	!Assign elements to matrix.!

!\class{ostream} !&!\func{operator}!$<<$(!\class{ostream}! &, !\templaten{matrix}!)
	!Print matrix.!

!\class{ostream} !&!\func{operator}!$<<$(!\class{ostream}! &, !\templaten{vector}!)
	!Print vector.!

\end{lstlisting}


\section{Constants}
\label{constants}

In addition to the normal constants there are close to one hundred indexing 
objects that may be used to subscript matrices, these are documented in 
\bref{indexing}.

\begin{lstlisting}{}
!\conceptn{Initializer} \const{zero}!
	!All zeros. $\mathbf 0$.!

!\conceptn{Initializer} \const{identity}!
	!Identity matrix. $\I$.!

!float \const{pi}!
	!Pi. $\pi$.!

!float \const{ln\_2}!
	!Natural logarithm of 2. $\log 2$.!

!float \const{epsilon}!
	!Same as! std::numeric_limits<float>::epsilon()

\end{lstlisting}

\section{Indexing}
\label{indexing}

It is possible to provide an alternate view of the data stored in a matrix 
without performing actual calculations on the data.
This is done by altering the element index calculation rather than altering
the data at the location.

Custom indexing modes can be created by making a new type that implements
\conceptn{Indexer}.

Aliasing between elements is detected and handled to eliminate repeated
calculations in all indexing modes except concatenation.
An aliased submatrix may not be used as an lvalue.

\subsection{Primitive transformations}

\subsubsection{Element selection}
A $1 \times 1$ matrix representing the element at the designated position is 
generated.
Note that a $1 \times 1$ matrix behaves like a scalar.

There are indexing objects for all of the named elements --- \id x, \id y,
\id z, \id w, \id s, \id t, \id u, \id v, \id r, \id g, \id b, \id a.

\subsubsection{Transpose}
Transposing $A$ generates $B = A^{\rm t}$, where $B_{ij} = A_{ji}$.

There is a transpose indexing object \id{tr}, but the function should
be preferred for readability.
Using the indexing object is the only way to get a transposed
\conceptn{LMatrix}.

\subsubsection{Cofactor}
The cofactor of an element $A_{ij}$ is the matrix formed by removing
row $i$ and column $j$ from $A$.

There are no indexing objects for cofactoring, use the function instead.
There is currently no way to get a cofactored \conceptn{LMatrix},
the feature is believed to be useless but could trivially be added.

\subsubsection{Submatrix selection}
Submatrices are selected using two ranges, $R = [r_1,r_2]$ and $C = [c_1,c_2]$.
A submatrix is formed from all elements $A_{ij}$ in the source matrix
where $i \in R$ and $j \in C$.

The ranges may be \emph{restricted}.
In a restricted range all positions in the range have to exist in
the matrix.
In a non-restricted range the largest subrange for which all positions 
are valid is used.

Indexing objects are provided for selecting specific rows and columns,
any submatrix rooted at $A_{11}$ and any row or column range.
Note that any submatrix can per definition be selected by concatenating 
a row- and a column range selection.

\id r$i$ = row $i+1$. \id c$i$ = column $i+1$.

\id m$ij$ = $A_{rc}$ where $(r,c) \in [1, i][1, j]$.

\id r$ij$ = all rows $\in [i+1, j+1]$.
\id c$ij$ = all columns $\in [i+1, j+1]$.

\subsubsection{Swizzle}
Swizzling is the most general primitive indexing transformation.
It is currently only supported for vectors.

A vector swizzling is described by a sequence of row indices
$I = (i_1, i_2 \dots)$.
The currently supported set of swizzlings is $\N^d$, where $d \in [2,4]$.
The element selection indexing transformation is more generic and
easier to use than $N^1$ swizzling would be and matrices with either
side larger than 4 are not completely supported anyway.
Aliasing between elements is allowed for rvalues.
%
Applying a $\N^d$ swizzling $I$ to a vector $v$ produces a $\R^d$ vector
$u$ where $u_j = v_{i_j}$.

There are indexing objects with coordinate names for all swizzlings
where $i_{j+1} > i_j$, i.e.\ \id{xy}, \id{su}, \id{xw} \dots
\ \id{stu}, \id{xyw} \dots.
Using color names there are objects for conversion between color storage
formats. \id{rgb}, \id{bgra}, \id{argb} \dots.
The combinations that have no names can be created by stacking element
selectors.

\subsection{Composite transformations}

\subsubsection{Concatenation}
Indexing transformations may be concatenated using the comma or application
operators.
Aliasing is not handled when concatenation is used.

The effect of a concatenated transformation $T = U V$ is that of
applying first $U$, then $V$. $A_T = A_{UV} = (A_U)_V$.

\subsubsection{Stacking}
Two general submatrices described by indexing objects, may be stacked
on top of each other to form a larger submatrix.

Applying a stacking transformation $T = (U, V)$ to a matrix $A$ yields
\[A_T = 
\left[\begin{array}{c}
	A_U \\
	A_V
\end{array}\right]\]

Stacking is applied using the bitwise or operator.
Stacking element selectors has the same effect as swizzling.


%\newpage
\appendix

\section{Understanding \mathlib{}}
\label{understanding}

\subsection{Delayed evaluation}
\label{exprtempl}

In order to minimize the use of temporaries in expressions
\mathlib{} uses delayed evaluation for its operations ---
the evaluation of operations is delayed until the final assignment.

Instead of returning a \templaten{matrix} most operations on matrices return
a closure object with information about the operation encoded in its type.%
\footnote{Closure objects with information encoded in the type are sometimes 
called \emph{expression templates}.}
The closure objects implement \conceptn{Matrix} and are virtually
indestiguishable from a \templaten{matrix} containing the result of the
operation.

As an example, the type of $A v + u$ is 
\template{tmatrix}{\template{matrix\_binop}{binop\_add, 
\template{matrix\_binop}{binop\_mmul$<$4$>$, \template{basic\_matrix}{4, 4}, 
\template{basic\_matrix}{4, 1}}, \template{basic\_matrix}{4, 1}}}.

In order to evaluate the expression we start decomposing the expression tree.
If there is an evaluation function for the current expression it is used,
otherwise the operands are evaluated.
The expression is then recombined with the same operation but with
completely evaluated arguments.

\subsection{Elementwise evaluation}
The closure objects described in \bref{exprtempl} provides information
about the entire expression.
This information may be used to evaluate a larger part of the expression
than simply the root.

An important case where this is possible is in expressions where the elements 
of the resulting matrix may be evaluated one at a time without having to
perform any nontrivial reevaluation of elements in the subexpressions.

For example the expression $(v a + u b) / c$ is evaluated using no temporaries
larger than a \lstinline!float!.

\subsection{Aliasing}
The delayed evaluation mechanism implies that some part of the right hand side
of an assignment may be overwritten during evaluation of the assignment.
%
Because of this, aliasing between the left and right side of an assignment is
not allowed in \mathlib.

The restriction against aliasing together with the techniques described
above is what makes \mathlib{} so fast, some inconvenience is the price
you pay.

Despite the general restriction, all elementwise expressions where 
$(A\ op\ B)_{ij} = A_{ij}\ op\ B_{ij}$ are unaffected by the aliasing problem.
This includes any combination of addition, subtraction, division and non-matrix
multiplication.

\subsection{SIMD operations}
Single Instruction Multiple Data (SIMD) specialized instructions
are supported when using GCC 3.2 with SSE instructions enabled
and when using Visual C++ 7.1 and optimizing for Pentium 4 or better.

When SIMD specializations are used it is important to minimize data
movement between registers and memory.
Because of this, it is usually better to perform vector operations
on entire vectors even if only the results for some of the elements are
actually needed.

The instruction usually used to copy vectors, \texttt{movaps},
assumes that all memory references are 16 byte aligned.
That assumption may not always be valid --- the runtime library
may not be able to provide appropriately aligned stack or heap objects.
A properly aligned stack may be inaproppriately offset by externally compiled
code if your code is used as a callback.
These issues must be considered when using SIMD specializations.

If the alignment requirements cannot be met, SIMD support may be disabled
by defining \texttt{MATHLIB\_NO\_SIMD}.

%Most of the the SIMD specializations for vectors are not subject to 
%delayed evaluation, 


\subsection{Better sorry than safe}
``Better sorry than safe, unless being safe is free'' is the design
philosophy of \mathlib.
The implications of this is that there are no runtime correctnes checks
for special cases or errors.
In particular, the restriction against aliasing is not enforced
--- sometimes aliased expressions will be evaluated correctly, sometimes 
they will not.

There are also cases where compile time checks could be used, to detect
element aliasing when using concatenated indexing modes for instance.
This has not been done because it would incur a significant increase in
compile time.

Since there is no way for a program to issue compile time warnings from a 
template, potentially dangerous constructs are silently accepted.

%\subsection{How smart is \mathlib?}
%Most cases where elementwise evaluation can be used are detected and handled
%automatically.

%\mathlib{} does \emph{not} handle cases where an expression could be
%evaluated more efficiently if the operands were reordered.

%One should remember that the worst case for \mathlib{} is roughly
%equivalent to the normal case for naive code.

\subsection{Requirements}
Ideally, \mathlib{} would be very portable since is does not require any 
special features from the compiler or environment but. In real life, few
compilers are yet good enough at handling templates to support \mathlib{}.

Performance may be increased if the compiler supports unaliased references.
Performance is greatly increased if the compiler and platform support
SIMD vector operations.

\begin{description}
\item[GCC] is the compiler mostly used to develop \mathlib{} and is thus
	fully supported. The SIMD specializations are only supported
	on GCC 3.1 and later. Just about any version of GCC should
	be able to compile \mathlib{} with a few modifications.
	Recent versions (3.2 or better) work as-is.
\item[Visual C++]
	Visual C++ .NET 2003 (7.1) is the first version conformant
	enough to compile \mathlib{}. Don't bother trying with
	earlier versions. SIMD is supported.
\end{description}

Please report results from other compilers to 
$<$wesslen@users.sourceforge.net$>$.

\section{Concepts}
\label{concepts}

In the concept descriptions all \lstinline!static!s are also \lstinline!const!s.

\subsection{Matrix}
\label{matrix_c}

\begin{lstlisting}{matrix_c}
concept !\concept{Matrix}{Rows, Cols}! : !\template{tmatrix}{Rows, Cols, Base}! {
	!\tpar{Base} has to implement \conceptn{MatrixBase}.!
}
\end{lstlisting}

\subsection{Vector}
\label{vector_c}

\begin{lstlisting}{vector_c}
concept !\concept{Vector}{Rows}! : !\concept{Matrix}{Rows, 1}! {
}
\end{lstlisting}


\subsection{LMatrix}

A \conceptn{LMatrix} is a \conceptn{Matrix} that can be assigned to.

\begin{lstlisting}{lmatrix_c}
concept !\concept{LMatrix}{Rows, Cols}! : !\concept{Matrix}{Rows, Cols}! {
	static bool is_lvalue = true;
}
\end{lstlisting}


\subsection{LVector}

A \conceptn{LVector} is a \conceptn{Vector} that can be assigned to.

\begin{lstlisting}{lvector_c}
concept !\concept{LVector}{Rows}! : !\concept{Vector}{Rows}! {
	static bool is_lvalue = true;
}
\end{lstlisting}

\subsection{MatrixBase}
\label{matrix_base_c}

\begin{lstlisting}{matrix_base_c}
concept !\conceptn{MatrixBase}! {
	static int rows, columns; !The size of the matrix.!
	static bool element_op; !If the matrix is a candidate for elementwise evaluation.!
	static bool is_storage; !If the elements of the matrix may be accessed without performing a calculation.!
	static bool is_lvalue; !If the matrix can be assigned to.!
	static bool error; !If there is an error in the matrix.!
	typedef base_type; !The type itself.!
	template<int i, int j> float element() const; !Read element.!
}
\end{lstlisting}
If the matrix is to be used as an \conceptn{LMatrix} it also requires
\begin{lstlisting}{matrix_base_c}
	template<int i, int j> float &element(); !Read/write element.!
\end{lstlisting}

\subsection{Indexer}
\label{indexer_c}

\begin{lstlisting}{indexer_c}
concept !\conceptn{Indexer}! : !\template{matrix\_indexer}{I}! {
	!\template{index\_info}{I, T} has to implement \conceptn{IndexDescriptor}.!
}
\end{lstlisting}

\subsection{IndexDescriptor}
\label{index_descriptor_c}

\begin{lstlisting}{index_descriptor_c}
concept !\concept{IndexDescriptor}{I, Base}! {
	static int rows, columns; !The size of the resulting matrix.!
	static bool aliasing; !If the indexer causes self-aliasing.!
	static bool error; !If there is an error in the indexer.!
	static unsigned element_mask; !Bitmask indicating which elements from the original matrix are extracted by the indexer.!

	template<int R, int C> struct index {
		static int r, c; !Original location representing R, C.!
	}
}
\end{lstlisting}


\section{Examples}
\label{examples}

\subsection{Basics}
\begin{lstlisting}{language=C++}
#include <math/matrix.h>
	// !Include matrix classes for declarations. Not needed if the line below is used.!
#include <math/matrix.tpp> // !Include matrix classes for usage.!

using namespace math; // !Use the \mathlib{} namespace.!

matrix<4> A, B, C;	// $A_{4 \times 4}$, $B_{4 \times 4}$, $C_{4 \times 4}$
matrix<3,2> D;		// $D_{3 \times 2}$
vector<4> v;		// $v \in \R^4$

A = identity;		// $A \leftarrow \I$
A(0,2) = 1;		// $A_{13} \leftarrow 1$

v = zero;		// $v \leftarrow \mathbf 0$
v.x = 1;		// $v_{\rm x} \leftarrow 1$
v[x] = 1;		// $v_{\rm x} \leftarrow 1$
v(0) = 1;		// $v_{\rm x} \leftarrow 1 \Leftrightarrow v_1 \leftarrow 1$

A = B * C;		// $A \leftarrow BC$
A *= B;			// $A \leftarrow AB$
\end{lstlisting}

\subsection{Indexing}
\begin{lstlisting}{language=C++}
transpose(A)		// $A^{\rm t}$
v[xz]			// $v_{\rm xz}$
v[a|g|r|b]		// $v_{\rm agrb}$ !Custom swizzling thru stacking.!
v[x|x]			// $v_{\rm xx}$ !Element aliasing is allowed.!
A[r2]			// !Row 3 of $A$.!
A[c0]			// !Column 1 of $A$.!
A[r0,c1]		// $A_{12}$
A[r12,c12]		// !$2 \times 2$ submatrix from the center of $A$.!
\end{lstlisting}


\section{License}
\mathlibr{} was placed in public domain 2003 by Daniel Wessl�n \\
$<$wesslen@users.sourceforge.net$>$

\end{document}
